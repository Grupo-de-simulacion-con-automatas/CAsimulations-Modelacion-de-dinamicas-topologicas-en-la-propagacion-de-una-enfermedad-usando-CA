%\newpage
%\setcounter{page}{1}
\begin{center}
\begin{figure}
\centering%
\includegraphics[scale=0.4]{Imagenes/logo_1.png}
%{file=uc.jpg,scale=1}%
\end{figure}
\thispagestyle{empty} \vspace*{2.0cm} \textbf{\huge
Predicción del comportamiento de una enfermedad simulada en autómatas celulares con un algoritmo propuesto en redes neuronales}\\[6.0cm]
\Large\textbf{Jorge Andrés Ibáñez Huertas}\\[4.0cm]
\small Universidad Central\\
Departamento de Matemáticas\\
Bogotá, Colombia\\
2021\\
\end{center}

\newpage{\pagestyle{empty}\cleardoublepage}

\newpage
\begin{center}
\thispagestyle{empty} \vspace*{0cm} \textbf{\huge
Predicción del comportamiento de una enfermedad simulada en autómatas celulares con un algoritmo propuesto en redes neuronales}\\[3.5cm]
\Large\textbf{Jorge Andrés Ibáñez Huertas}\\[3.0cm]
\small Trabajo de grado presentado como requisito parcial para optar al
t\'{\i}tulo de:\\
\textbf{Matemático}\\[3.0cm]
Director:\\
Carlos Isaac Zainea\\[3.5cm]
Universidad Central\\
Departamento de Matemáticas\\
Bogotá, Colombia\\
2021\\
\end{center}

\newpage{\pagestyle{empty}\cleardoublepage}

% \newpage
% \thispagestyle{empty} \textbf{}\normalsize
% \\\\\\%
% \textit{"Tan sólo por la educación puede el hombre llegar a ser hombre.\\ El hombre no es más que lo que la educación hace de él"\\
% \textit{Immanuel Kant}}\\[4.0cm]

\begin{flushright}
\begin{minipage}{8cm}
    \noindent
        \small
\end{minipage}
\end{flushright}

\newpage{\pagestyle{empty}\cleardoublepage}

\newpage
\thispagestyle{empty} \textbf{}\normalsize
\\\\\\%
\textbf{\LARGE Agradecimientos}\\

% Agradezco a mis padres Roosvelt Rivera y Angelica Useche por los años de apoyo incondicional en mis años de estudio, además de la profesora Edel María Serrano Iglesias (Q.E.P.D) sin la cuál no hubiera entrado al mundo de las matemáticas.\\
% También agradezco a los profesores que a lo largo de la carrera estuvieron dispuestos a enseñar y guiar con incontable paciencia y dedicación, entre ellos: Xiomara Rojas, Diana Herrera, Diana Pulido, Isaac Zainea, Miguel Pachón, Henry Naranjo, Fabián Sánchez, Nicolás Avilán y mi director de tesis Henry Sánchez.

\newpage{\pagestyle{empty}\cleardoublepage}

\newpage
\textbf{\LARGE Resumen}\\

% En este trabajo se desarrollan los resultados obtenidos en el artículo  de Morales \& Arbieto,  específicamente la distancia $GH^0$ dada en ese manuscrito. Una vez se haya comprendido la fundamentación y  analizado algunas propiedades de dicha métrica, se procederá a estudiar la estabilidad topológica de las dinámicas sobre espacios compactos utilizando en principio la noción de Walters con una métrica $C^0$. Luego, se define y analiza la noción de $GH$-estabilidad  resaltando la relación que existe entre ambas nociones.\\

\textbf{\LARGE Abstract}\\\\

% In this work the results obtained in the article by Morales \& Arbieto are developed, specifically, the distance $ GH^0$ given in the article. Once the foundation is understood and some properties of that metric have been analyzed, we will proceed to study the topological stability of the dynamics on compact spaces using a notion given by Walters with a  $C^0$-metric. After that, will be defined and analyzed the $GH$-stability notion highlighting the relationship between both notions.
%\newpage
%\setcounter{page}{1}
\begin{center}
\begin{figure}
\centering%
\includegraphics[scale=0.4]{Imagenes/logo_1.png}
%{file=uc.jpg,scale=1}%
\end{figure}
\thispagestyle{empty} \vspace*{2.0cm} \textbf{\huge
Predicción del comportamiento de una enfermedad simulada en autómatas celulares con un algoritmo propuesto en redes neuronales}\\[6.0cm]
\Large\textbf{Jorge Andrés Ibáñez Huertas}\\[4.0cm]
\small Universidad Central\\
Departamento de Matemáticas\\
Bogotá, Colombia\\
2021\\
\end{center}

\newpage{\pagestyle{empty}\cleardoublepage}

\newpage
\begin{center}
\thispagestyle{empty} \vspace*{0cm} \textbf{\huge
Predicción del comportamiento de una enfermedad simulada en autómatas celulares con un algoritmo propuesto en redes neuronales}\\[3.5cm]
\Large\textbf{Jorge Andrés Ibáñez Huertas}\\[3.0cm]
\small Trabajo de grado presentado como requisito parcial para optar al
t\'{\i}tulo de:\\
\textbf{Matemático}\\[3.0cm]
Director:\\
Carlos Isaac Zainea\\[3.5cm]
Universidad Central\\
Departamento de Matemáticas\\
Bogotá, Colombia\\
2021\\
\end{center}

\newpage{\pagestyle{empty}\cleardoublepage}

% \newpage
% \thispagestyle{empty} \textbf{}\normalsize
% \\\\\\%
% \textit{"Tan sólo por la educación puede el hombre llegar a ser hombre.\\ El hombre no es más que lo que la educación hace de él"\\
% \textit{Immanuel Kant}}\\[4.0cm]

\begin{flushright}
\begin{minipage}{8cm}
    \noindent
        \small
\end{minipage}
\end{flushright}

\newpage{\pagestyle{empty}\cleardoublepage}

\newpage
\thispagestyle{empty} \textbf{}\normalsize
\\\\\\%
\textbf{\LARGE Agradecimientos}\\

El trabajo realizado en la presente tesis no habría comenzado sin la guía y los consejos de mi tutor Isaac Zainea y del profesor Nicolas Avilán. A ellos les doy mi más sincero agradecimiento por escuchar mis ideas y ayudarme a materializarlas. Por haberme introducido en las ciencias de la computación, el modelado y la simulación de eventos reales. \\

Quiero agradecer también a los profesores del departamento de matemáticas, en particular al profesor Henry Sánchez por sus consejos durante el desarrollo de esta tesis. Asimismo estoy muy agradecido con las profesoras Diana Pulido, Xiomara Rojas y Diana Herrera por sus consejos en mis años de estudio. Agradezco también a la profesora Edel Maria Serrano Iglesias (Q.E.P.D), ya que sin ella no hubiera entrado al mundo de las matemáticas. \\

Por último, agradezco a mi familia, en particular a mis padres Alexandra Huertas y Jorge Ibáñez que siempre han estado a mi lado de manera incondicional. Fue gracias a su apoyo y sacrificio que pude llegar tan lejos.

\newpage{\pagestyle{empty}\cleardoublepage}

\newpage
\textbf{\LARGE Resumen}\\

% En este trabajo se desarrollan los resultados obtenidos en el artículo  de Morales \& Arbieto,  específicamente la distancia $GH^0$ dada en ese manuscrito. Una vez se haya comprendido la fundamentación y  analizado algunas propiedades de dicha métrica, se procederá a estudiar la estabilidad topológica de las dinámicas sobre espacios compactos utilizando en principio la noción de Walters con una métrica $C^0$. Luego, se define y analiza la noción de $GH$-estabilidad  resaltando la relación que existe entre ambas nociones.\\

\textbf{\LARGE Abstract}\\\\

% In this work the results obtained in the article by Morales \& Arbieto are developed, specifically, the distance $ GH^0$ given in the article. Once the foundation is understood and some properties of that metric have been analyzed, we will proceed to study the topological stability of the dynamics on compact spaces using a notion given by Walters with a  $C^0$-metric. After that, will be defined and analyzed the $GH$-stability notion highlighting the relationship between both notions.
\chapter{Introducción}\label{ch;Introduccion}

La predicción del comportamiento de una enfermedad, su nivel de afectación en una población y las maneras de controlarla son los aspectos más importantes que se estudian en la epidemiología por medio de herramientas como datos históricos y modelos matemáticos.

Problemáticas como las ocasionadas por enfermedades como la gripe, la viruela, el VIH y más recientemente el Covid-19 han motivado el desarrollo de una gran variedad de modelos aplicados a diferentes enfoques dentro del estudio epidemiológico. Algunos ejemplos particulares son el nivel de propagación considerando los patrones de movilidad dentro de una región \cite{colaGNN, epidemiologicalNeuralNetwork}, el impacto de medidas como el aislamiento preventivo para la disminución de contagios \cite{stayHome}, la vacunación de la población \cite{shortHistory}, los contactos de individuo a individuo \cite{heterogeneousPopulation}, las relaciones entre individuos \cite{redesComplejas} y las interacciones en masa \cite{combiningGraph, transfer2021} que sirven como punto de partida para generar pronósticos sobre los comportamientos de enfermedades como la gripe, la viruela o incluso enfermedades de transmisión sexual como el VIH en una población determinada.

Si bien la predicción de la propagación de una enfermedad es un problema que se ha visto desde varios puntos y con distintas herramientas matemáticas y computacionales, hemos identificado dos grandes grupos de modelos: 
\begin{enumerate}
    \item El primero tenemos a los modelos compartimentales clásicos. Los cuales emplean mecanismos de progresión de la enfermedad para describir dinámicas en una población, a partir de sistemas de ecuaciones diferenciales como el modelo SIS, SIR, MSEIR, etc. Estas ecuaciones se definen por medio de reglas lógicas a partir de supuestos sobre comportamientos individuales, como por ejemplo, el cambio de estado debido al contacto entre individuos de diferentes compartimientos.
    
    En este grupo también podemos encontrar a los modelos basados en agentes, los cuales a menudo se soportan en los modelos compartimentales clásicos para analizar comportamientos demográficos a partir de simulaciones de la propagación de la enfermedad entre los individuos. Usualmente, este tipo de modelos se implementa sobre autómatas celulares y/o redes complejas. Las reglas para desarrollar este tipo de modelos se fundamentan en gran parte por reglas lógicas sobre las dinámicas individuales, teniendo presente la fundamentación teórica de diversos campos como la teoría de grafos, las redes complejas, la lógica, el cálculo y la teoría de sistemas dinámicos.
    \item En el segundo grupo encontramos los modelos que implementan técnicas de aprendizaje como las redes neuronales, algoritmos de clasificación, redes neuronales basadas en grafos, entre otros. A pesar de que estos modelos son capaces de generar pronósticos con una amplia aplicabilidad, poseen un problema importante que tiene que ver con la obtención de los datos necesarios para alimentar los diferentes procesos de aprendizaje y en ocasiones, apenas se tienen en cuenta los efectos espaciales lo que dificulta su aplicabilidad en escenarios a largo plazo.
\end{enumerate}

Los datos juegan un papel primordial al momento de generar pronósticos realistas y así poder implementar medidas de prevención y control sobre la enfermedad en cuestión. Ejemplos de esto son los modelos analizados en \cite{epidemiologicalNeuralNetwork, combiningGraph, forecasting} y en \cite{transfer2021}, en los que a partir de algoritmos de clasificación y redes neuronales basadas en grafos se analizan las dinámicas al rededor del Covid-19.

Sin embargo, en la mayoría de ocasiones los datos están mal tomados o simplemente no existen. Para confrontar este tipo de limitaciones usualmente se simulan datos a partir de comportamientos observados en la población o en eventos anteriores, como es en el caso de \cite{populationDensity}, en donde a partir de una serie de medidas de movilidad entre regiones de la población en Polonia se simula la propagación de una enfermedad usando autómatas celulares.

Los autómatas celulares son una herramienta con una aplicabilidad particularmente amplia en los modelos epidemiológicos debido a los comportamientos globales que pueden ser generados a partir de comportamientos locales, la generación de datos fácilmente interpretables y su capacidad de implementar nuevas características como en \cite{spatialDependences, populationDensity, globalStochastic}.

Hemos evidenciado la inexistencia de un algoritmo capaz de realizar predicciones para el comportamiento de una enfermedad que considere las interacciones de individuo a individuo. Esto se debe a la naturaleza con la que se almacenan los datos de la misma enfermedad como número de contagios, muertes causadas por la enfermedad, entre otros generando así, predicciones limitadas por los comportamientos cercanos entre los individuos. 

Teniendo en cuenta la aplicabilidad de los autómatas celulares y las cualidades predictivas de los modelos en redes neuronales, nos podemos plantear el objetivo de diseñar un algoritmo en redes neuronales que permita realizar pronósticos sobre el comportamiento de una enfermedad simulada con autómatas celulares, teniendo en cuenta aspectos topológicos que modelen las interacciones entre individuos para responder a la pregunta: ¿Qué impacto tienen las relaciones sociales cercanas, en la propagación de una enfermedad?

Para responder a nuestra pregunta generadora nos propusimos las siguientes fases metodológicas:
\begin{enumerate}
    \item Modelos basados en autómatas celulares para la simulación de una enfermedad.
    \item Análisis sobre la simulación de la enfermedad y comparación con modelos compartimentales clásicos.
    \item Diseño e implementación del algoritmo en redes neuronales sobre datos simulados.
    \item Validación del algoritmo frente a cambios de topología.
\end{enumerate}

En donde la primera fase tendrá una etapa de comprensión sobre los modelos epidemiológicos clásicos en ecuaciones diferenciales y la teoría de autómatas celulares para posteriormente desarrollar el algoritmo que permita simular el comportamiento de la enfermedad usando los autómatas celulares. Nos enfocaremos puntualmente sobre los modelos SIS y SIR en sus variaciones que consideran la muerte causada por la enfermedad.

En la segunda fase nos encontraremos en una etapa de validación del modelo planteado en autómatas celulares con respecto a los modelos clásicos. Esta fase nos permitirá establecer las ventajas que posee el modelo que proponemos con respecto a los modelos en ecuaciones diferenciales, para esto será necesario implementar una forma de visualizar los datos obtenidos por la simulación en autómatas celulares que permita realizar la comparación con los datos generados por los modelos clásicos. Adicionalmente presentamos una visualización espacial del comportamiento de la enfermedad en las células de nuestro modelo.

La fase 3 inicia con una etapa de investigación sobre redes neuronales y sus aplicaciones a problemas similares. Posteriormente construiremos un algoritmo en redes neuronales sobre los datos generados por la simulación en autómatas celulares (AC), para lo cual será necesario diseñar una forma de obtener los datos para el entrenamiento del algoritmo, definir una función objetivo y establecer una métrica para el error de nuestro algoritmo con respecto a los resultados de diferentes simulaciones de enfermedades en AC.

Para la última fase de nuestro proyecto será necesario conocer y comprender los conceptos de topología que giran en torno a los sistemas de vecindades, esto con el objetivo de implementar una herramienta que permita generalizar el algoritmo de la fase 1 de manera que se tengan en cuenta diferentes topologías y puntualmente, diferentes sistemas de vecindades locales. Esto nos permitirá ampliar la aplicabilidad del modelo propuesto, ya que se podrán modelar diferentes tipos de interacciones entre células a partir de los sistemas de vecindades locales. Finalmente aplicaremos el algoritmo de la fase 3 para conocer su efectividad frente a cambios de topología.

El presente documento comienza con el capítulo de preliminares, en donde se exponen los análisis sobre los modelos epidemiológicos clásicos (sección 2.1), los conceptos de topología y autómatas celulares usados para la simulación de la enfermedad (secciones 2.2 y 2.3) y finalmente, los conceptos que usaremos para la construcción de nuestro algoritmo en redes neuronales (sección 2.4).

En el capítulo 3 se presenta la manera en la que se modela la interacción entre células a partir de los sistemas de vecindades fundamentales para luego definir las reglas de evolución que modelaran los comportamientos de una enfermedad considerando tres posibles variaciones: la versión simple en la que no se considera la muerte o el nacimiento de las células; la versión con natalidad y mortalidad, en la que se presenta una variación con respecto a los modelos clásicos y finalmente la versión que considera la muerte causada por la enfermedad.

